\documentclass[12pt]{article}
\usepackage[paper=a4paper,left=25mm,right=25mm,top=25mm,bottom=25mm]{geometry}
\usepackage[english]{babel}
\usepackage[utf8]{inputenc}
\usepackage[pdftex]{graphicx}
\usepackage{tikz-qtree}
\usepackage{color}
\usepackage{amssymb}
\usepackage{amsthm}
\usepackage{hyperref}
\usepackage{enumitem}
\usepackage{pdfpages}
\usepackage{hyperref}


\linespread{1.25}

\begin{document}
\tikzset{every tree node/.style={minimum width=2em,draw,circle},
         blank/.style={draw=none},
         edge from parent/.style=
         {draw,edge from parent path={(\tikzparentnode) -- (\tikzchildnode)}},
         level distance=1.5cm}
\begin{titlepage}
\includegraphics[height=20mm]{images/uzh_logo}\\

\begin{flushleft}

\vspace{2cm}

{\Large Introduction to Artificial Intelligence\\Exercise Sheet 3}\\

\vspace{4cm}

\textbf{Laurin van den Bergh, 16-744-401\\Yufeng Xiao, 19-763-663\\Nora Beringer, 19-734-227}\\

\vspace{2cm}

Universität Zürich\\
Institut für Informatik

\vfill Due: March 16, 2022

\vspace{3cm}


\end{flushleft}
\end{titlepage}

\newpage

\section*{Exercise 3.1}
a) The wolf and goat can never be alone on a river bank, as well as the cabbage and goat can never be alone on a river bank.\\ Heuristic: number of items on the wrong river bank, except the farmer.\\
Initial state: $\langle L, L, L, L \rangle$ \\
Goal state: $\langle R, R, R, R \rangle$ \\
The goal state is reached in at least three steps, as wolf, goat and cabbage need to be placed separately on the goal river bank. We ignored states which will lead to a sink state. The result is informative in the way that we know that reaching the goal state can be achieved in at least three steps. \\
b) It is admissible as it doesn't overestimate the cost of reaching the goal state as the lowest possible cost from the initial state is at least three steps. Our heuristic counts the number of items on the left river bank. To reach a goal state each item has to be moved to the right river bank. Therefore our heuristic provides a lower bound on the number of necessary moves. \\Side note: If the heuristic is admissible, then it is safe and goal-aware. \newline
c) It is consistent as the estimated cost of reaching the goal from the initial state is no greater than the step cost of getting to the successor state (crossing the river) plus the estimated cost (at least three steps) of reaching the goal state from the initial state. \\ As the action of crossing the river only decreases the items which are left behind on the left river side only by one during each move, it never goes down below the estimated lower bound of at least 3 moves. \\Side note: If the heuristic is goal-aware and consistent, then it is admissible. \newline



\section*{Exercise 3.2}

a) Initial state = head node: Basel \\
b)   \\
c) \newline


\section*{Exercise 3.3}

Programming




\end{document}