\documentclass[12pt]{article}
\usepackage[paper=a4paper,left=25mm,right=25mm,top=25mm,bottom=25mm]{geometry}
\usepackage[english]{babel}
\usepackage[utf8]{inputenc}
\usepackage[pdftex]{graphicx}
\usepackage{color}
\usepackage{amssymb}
\usepackage{amsthm}
\usepackage{hyperref}
\usepackage{enumitem}
\usepackage{pdfpages}

\linespread{1.25}

\begin{document}
\begin{titlepage}
\includegraphics[height=20mm]{images/uzh_logo}\\

\begin{flushleft}

\vspace{2cm}

{\Large Introduction to Artificial Intelligence\\Exercise Sheet 0}\\

\vspace{4cm}

\textbf{Laurin van den Bergh, 16-744-401\\Yufeng Xiao, 19-763-663\\Nora Beringer, 19-734-227}\\

\vspace{2cm}

Universität Zürich\\
Institut für Informatik

\vfill Due: February 25, 2022

\vspace{3cm}


\end{flushleft}
\end{titlepage}

\newpage

\section*{Exercise 0.2 - Literature}

\subsection*{First sentence Chapter 11.1 - Definition of Classical Planning:}

"Classical planning is defined as the task of finding a sequence of actions to accomplish a goal in a discrete, deterministic, static, fully observable environment."

\end{document}